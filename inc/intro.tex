\section{Introduction}
\label{sec:intro}
    Inverse pyramid:
    \begin{itemize}
        \item P1: Background
        \begin{enumerate}
          \item Inverse pyramid
          \item Last sentence: introduce the area of the title of your paper as a significant topic
          \item 5 refs (books, review papers, journals, no conf)
        \end{enumerate}
        \item P2-3: Background
        \begin{enumerate}
          \item Original classification: Support the objectives with each sentence
          \item Appreciate their work, their focus however different
          \item no technical words
          \item 10 refs (journal, conference papers)
        \end{enumerate}
        \item P4: Objectives
        \begin{enumerate}
          \item 1-3 objectives
          \item the first sentence: this paper presents
        \end{enumerate}
        % \item P5: Outline
    \end{itemize}

    The organization of the paper as follows.
    \Cref{sec:fundamentals} lays out the fundamentals of the radio wave propagation, while \Cref{sec:mfms} covers the details of the proposed system.
    \Cref{sec:experimentation} demonstrates the validity of the proposed system in different environments.
    In \Cref{sec:conclusion}, the experimentation results will be concluded and future work will be addressed.
    % \item Start from importance of localization
    % \item importance of indoor localization
    % \item GPS-denied, that's why radio waves
    % \item Radio channels are time-varying, noisy and prone to different sources of failure
    % \item many solution has been proposed but my contributions are more significant
    % \item outline
