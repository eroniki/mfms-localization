\section{Fundamentals of Radio Wave Propagation in Radiolocationing}
\label{sec:fundamentals}
    This section will cover the fundamentals of radio wave propagation in regards to \gls{firl}.
    \Gls{firl}s are consisted of two main phases.
    The first phase, i.e.\ offline phase,  involves with collecting some form of measurements about the anchor nodes placed in the environment, with corresponding positions.
    Let $m^i_j \in \mathbb{Z}^1$ and $d^i_j \in \mathbb{R}^1$ to be the measurement obtained from anchor node $i$ at a location $j$, and the radial distance of between location $j$ and the anchor node $i$, respectively.
    The measurement space $\bm{m}^i=\{m^i_j | j=1 \ldots n_{loc}\} \in \mathbb{R}^{n_{node}}$ are often constructed in either frequency domain [refer CSI], or time domain.
    % The collected measurements are used to approximate the propagation function for each anchor node.
    The measurements are then used to approximate the propagation function $f^i_d$ of each anchor node, with which $\mathbb{Z}^1\mapsto\mathbb{R}^1$.

    \begin{equation}
        f^i_d(m^i_j)=d^i_j
    \end{equation}

    One of the most popular fingerprint is \gls{rss} in \gls{firl}s due to its simplicity in acquiring the fingerprints.
    The fundamental relationship between transmitted and received signal strengths $P_t$ and $P_r(d)$ occurred between ideal antennas in an empty space with a distance $d$ separation is characterized by \gls{ffse} given below~\cite{friis1946note}.

    \begin{equation}
        \label{eq:friisWatts}
            P_r(d) = \dfrac{P_t  G_t  G_r \lambda^2}{{\left(4 \pi d\right)}^2}
    \end{equation}

    % In \Cref{eq:friisWatts}, $P_r(d)$ and $P_t$ denote received, and transmitted signal strengths with $d$ meters separation between two antennas in Watts, respectively.
    In \Cref{eq:friisWatts}, $G_t$, and $G_r$ denote unitless gains of transmitter and receiver antennas, respectively, whereas $\lambda$ is the wavelength of the radio wave.
    However well-known, \gls{ffse} is not able to model propagation mechanisms, i.e.\ reflection, diffraction or scattering occurring along the the propagation path, due to the fact that it can only model \gls{los} propagation.
    Thus, it is often impractical to utilize it in indoor localization problems where reflection and diffraction occurs along the propagation path.
    Moreover, since the received signal strength is often minuscule level, \gls{ffse} is denoted in decibel scale relative to a milliwatt.

    \begin{equation}
      \begin{split}
        \label{eq:friisdBm}
        \widetilde{P}_r(d) &= \widetilde{P}_t + 10 \log{G_t} + 10 \log{G_r} + \\
        & 20 \log{\lambda} - 20 \log{d} - 20 \log{4 \pi}
      \end{split}
    \end{equation}
    $\widetilde{P}_r(d)$ and $\widetilde{P}_t$ represent received and transmitted signal strengths decibel scale.
    However, neither \Cref{eq:friisWatts}, nor \Cref{eq:friisdBm} holds true for the distance $0<d<\lambda$.
    Thus, received signal strength generally is denoted relative to a reference point $d_0$ with a prior corresponding received signal strength.

    \begin{equation}
        \label{eq:friisRef}
        \widetilde{P}_r(d) = \widetilde{P}_r(d_0) + 20 \log{\dfrac{d_0}{d}}
    \end{equation}

    % \begin{equation}
    %     \label{eq:pathloss}
    %     PL(d) = 10 \log{\dfrac{P_t}{P_r}}
    % \end{equation}

    Based on \gls{ffse}, path loss occurring along the propagation path can be derived as the difference between transmitted and received signal strength.
    \Cref{eq:log-distance} represents a special \gls{pl} model, i.e. \gls{ldpl} which describes the attenuation relative to a reference point.
    One of the major advantages of \gls{ldpl} model over \gls{ffse} is that \gls{ldpl} model can account for obstructions and corresponding wave propagation in the space with varying values of \gls{pl} exponent $n$.
    % \gls{pl} represents the attenuation occuring between receiver and transmitter antennas in decibel scale.

    % Along with representing the received signal strength with a reference point, \gls{pl} is another common terminology used in field.

    \begin{equation}
        \label{eq:log-distance}
        \overline{PL}(d) = \overline{PL}(d_0) + 10 n \log{\dfrac{d}{d_0}}
    \end{equation}

    \begin{equation}
        n =
        \begin{cases}
            <2,& \parbox[t]{.6\textwidth}{if the space structure guides the radio waves\\
                                        along the propagation path}, \\
            =2,& \text{if the space is empty}, \\
            >2,& \parbox[t]{.6\textwidth}{if there are obstructions along the propagation\\
                                        path}
        \end{cases}
    \end{equation}

    One of the most visited solution in \gls{firl}s is to estimate $n$ by fitting a curve to collected fingerprints.
    Given a mean \gls{pl} at an unknown location $\overline{PL}(d)$, \gls{pl} exponent $n$ and a mean \gls{pl} at the reference point $d_0$ $\overline{PL}(d_0)$, the propagation function of anchor node i $f^i_{d}$ can be obtained by solving \gls{ldpl} for the radial distance $d^i$.

    \begin{equation}
      \label{eq:log-distance-d}
      f^i_{d} = d^i_0 10^{\left(\dfrac{\overline{PL}(d)-\overline{PL}(d_0)}{10 n} \right)} = d^i_j
    \end{equation}


    However, in order to obtain radial distance from anchor node i $d^i$, \gls{pl} exponent $n$ should be estimated from the data collected during the offline phase.
    Let $\bm{m}^i = \{\widetilde{P}_r^{i,j}(d) | j=1 \ldots n_{loc} \}$ and $\bm{d}^i = \{ d^i_j | j = 1 \ldots n_{loc}\}$ be the \gls{rss} fingerprints acquired from anchor node $i$ during surveying and corresponding distances from anchor node $i$, respectively.
    The estimated radial distance from anchor node i $\hat{d}^i$ can be obtained with the approximated propagation function $\hat{f}^i_d(m^i_j, n_i^*)$.

    \begin{equation}
      \label{eq:log-distance-d-app}
      \hat{f}^i_d(m^i_j, n_i^*) = d^i_0 10^{\left(\dfrac{\widetilde{P}_t - \widetilde{P}_r^{i,j}(d) - \overline{PL}(d_0)}{10 n_i^*} \right)} = \hat{d}^i_j
    \end{equation}
    where $n^*$ is the overall \gls{pl} exponent which minimizes the absolute localization error $|d^i_j - \hat{f}^i_d(m^i_j, n^*)|$ where  $j = 1 \ldots n_{loc}$.

    \begin{equation}
      n^*_i = \argmin_n \bm{e}^i  = \argmin_n \{d^i_j - \hat{f}^i_d(m_i, n)\}
    \end{equation}

    After obtaining the approximated propagation function, the measurements acquired from the anchor nodes are mapped to relative radial distances in order to distinguish the absolute position of the agent in the environment, which forms the offline phase of the \gls{firl}s.
    % One of the most prominent method is trilateration where the intersection of the radial distances is assumed to be the absolute location of the agent.
    % However, this method is impratical in \gls{firl}s due to dynamic nature of the indoor environments, and high probability of receiving degraded signal caused by multipath or shadowing effects.
    % During offline phase, there exists many methods to acquire absolute location of the agent, namely, trilateration method [trilateration], least squares method [least squares], linear regression [linear regression].
