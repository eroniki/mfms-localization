\section{Fundamentals of Radio Wave Propagation in Radiolocationing and Function Approximation Methods}
    \label{sec:fundamentals}
    The first part of this section will first cover the fundamentals of radio wave propagation in regards to radiolocationing systems based on fingerprinting while the second part will present a brief description of function approximation methods with a emphasis on a stochastic function approximation method, i.e. \gls{gru}.
    %  in the context of radio wave propagation function.
    % First the radio wave propagation will be formulated and then the existing methods for radiolocation systems will be covered.

    \subsection{Radio Wave Propagation In Radiolocationing Systems}
        % An ideal receiving antenna in an empty space with a gain $G_r$ and placed $d$ meters away from a transmistting antenna with a gain $G_t$ and output po
        % An ideal antenna in an empty space with a gain of $G_t$ and radiating power of $P_t$, induces power $P_r$ in an antenna which has a gain of $G_r$
        The fundamental relationship between transmitted power $P_t$ and the received power $P_r$ occurred between ideal antennas in an empty space with a distance $d$ separation is characterized by Friis' Free Space Equation given below~\cite{friis1946note}.

        \begin{equation}
            \label{eq:friisWatts}
                P_r(d) = \dfrac{P_t  G_t  G_r \lambda^2}{{\left(4 \pi d\right)}^2}
        \end{equation}

        In \Cref{eq:friisWatts}, $P_r(d)$ and $P_t$ denote received, and transmitted power with $d$ meters separation between two antennas in Watts, respectively.
        $G_t$, $G_r$, and $\lambda$ represent unitless gains of transmitter and receiver antennas, and the wavelength of the radio wave, respectively.
        Since the received power is minuscule level, \Cref{eq:friisdBm} represents Friis' equation in dBm.

        \begin{equation}
          \begin{split}
            \label{eq:friisdBm}
            P^{+}_r(d) &= P^{+}_t + 10 \log{G_t} + 10 \log{G_r} + \\
            & 20 \log{\lambda} - 20 \log{d} - 20 \log{4 \pi}
          \end{split}
        \end{equation}
        $P^{+}_r(d)$ and $P^{+}_t$ represent received and transmitted powers decibel scale.
        However, neither \Cref{eq:friisWatts}, nor \Cref{eq:friisdBm} holds true for the distance $d = 0$ and $d<\lambda$.
        Thus, received power generally is denoted relative to a reference point $d_0$ with a prior corresponding received power.

        \begin{equation}
            \label{eq:friisRef}
            P^{+}_r(d) = P^{+}_r(d_0) + 20 \log{\dfrac{d_0}{d}}
        \end{equation}

        % \begin{equation}
        %     \label{eq:pathloss}
        %     PL(d) = 10 \log{\dfrac{P_t}{P_r}}
        % \end{equation}
        Along with representing the received power with a reference point, path loss is another common terminology used in field.
        The path loss $PL(d)$ represents the attenuation occuring between receiver and transmitter antennas in decibel scale.
        \Cref{eq:log-distance} represents a special path loss model, i.e. \gls{ldpl}.
        \gls{ldpl} describes the attenuation relative to a reference point.
        One of the major advantages of \gls{ldpl} model over Friis' free space model is that log-distance path loss model can account for different spaces by varying values of $n$.

        \begin{equation}
            \label{eq:log-distance}
            \overline{PL}(d) = \overline{PL}(d_0) + 10 n \log{\dfrac{d}{d_0}}
        \end{equation}

        \noindent where $n$ is the path loss exponent and varies depending on the environment.
        Please note $n = 2$ for empty spaces where there is no reflectors, diffractors or scatterers available in the propagation path.
                While the most visited solution in fingerprinting-based indoor radiolocationing system is to estimate $n$ by fitting a curve to collected fingerprints.
        % After acquiring $n$ is enough for solving for the radial distance $d$ from the anchor node, at least 4 anchor nodes are required to localize an agent in an environment.

        Given the mean path loss $\overline{PL}(d)$ at a location, the path loss exponent $n$ and the mean measurement $\overline{PL}(d_0)$ at the reference point $d_0$, the propagation function of anchor node i $f^i_{d}$ which maps measurements to radial distance $d$ can be written as below.
    % :\mathbf{m}^i \mapsto \mathbf{d}^i
        \begin{equation}
          \label{eq:log-distance-d}
          f^i_{d} = d_0 10^{\left(\dfrac{\overline{PL}(d)-\overline{PL}(d_0)}{10 n} \right)} = d^i
        \end{equation}


    \subsection{Function Approximation Methods}
        However, in order to obtain radial distance from anchor node i $d^i$, the path loss exponent $n$ should be estimated from the collected data.
        Let $\mathbf{m}^i = \{m^i_j | j=1 \ldots n_{loc} \}$ and $\mathbf{d}^i = \{ d^i_j | j = 1 \ldots n_{loc}\}$ be the fingerprints acquired from anchor node $i$ during surveying and corresponding distances from anchor node $i$, respectively.
        The estimated radial distance from anchor node i $\hat{d}^i$ can be obtained with the approximated propagation function $\hat{f}^i_d(m^i_j, n_i^*)$.

        \begin{equation}
          \label{eq:log-distance-d}
          \hat{f}^i_d(m^i_j, n_i^*) = d^i_0 10^{\left(\dfrac{P^{+}_t - m^i_j - \overline{PL}(d_0)}{10 n_i^*} \right)} = \hat{d}^i_j
        \end{equation}

        \noindent where $n^*$ is the overall path loss exponent which minimizes the absolute localization error $|d^i_j - \hat{f}^i_d(m^i_j, n^*)|$ where  $j = 1 \ldots n_{loc}$.

        \begin{equation}
          n^* = \argmin_n \mathbf{e}^i  = \argmin_n \{d^i_j - \hat{f}^i_d(m_i, n)\}
        \end{equation}

        Least square estimation, linear regresion, SVM, Neural Nets, genetic algorithm and bla bla bla.


    % Indoor localization is an important problem in which an object of interest, i.e.\ a robot in our framework, suited with different sensors localizes itself in an indoor environment where there is no global positioning information is available.
    % \textit{The complexity of the problem significantly \# \textit{exponentially} increases as NLoS of reference AP's, presence of hard-constraints, in particular infrasractural elements such as walls and doors, noisy nature of the signals, and dynamic environments.}~\cite{liu2007survey}
    % \textit{As robotic systems find more applications in indoor areas where dynamic objects, such as other robots and humans, often coexist, it increasingly becomes important to safely and accurately localize the agent.} %where there is no reliable global positioning system is available.}
    % % \textit{Maybe co-robots can be mentioned.}
    % Thus, a great amount of interest has been showed from both academia and industry.
    % \# \textit{Do I really emphasize industry academia, actually this is a good opportunity to mention iBeacon from Apple}
    % The indoor localization systems based WiFi signal can be mainly categorized undepngr two categories: fingerprinting and model-based methods~\cite{hossain2015survey}.
    % We'll, however, only cover the fingerprinting technique due to the increasing popularity of the technique.



    % \begin{itemize}
    %   \item CSI-related special \\
    %     hardware requirement
    %   \item Propagation-modelling \\
    %     multi-path effect difficult to model
    %   \item Fingerprinting \\
    %     An emerging area learning fingerprints is deep learning~\cite{gao2015channel}
    % \end{itemize}
